
\chapter{Introducción}

\section{Objetivos}

En este trabajo se explora el uso de modelos gráficos probabilísticos para el objetivo de predicción (clasificación supervisada) a partir de datos sobre las principales expresiones genómicas de las encimas asociadas a la ``vía de las kinureninas'' (KP, kynurenine pathway). Además, se exploran diferentes métodos (regresión logística, análisis de discriminante, maquina de soporte vectorial, etc.) para comparar los resultados en términos de poder predictivo.

El proyecto ``UCSC Xena'' \cite{Goldman} pone a disposición las bases de datos unidas sobre secuenciación genómica de diferentes proyectos para el análisis del cáncer.

Una de las bases de datos más usadas y que es la que se considera usar en este trabajo está disponible con el nombre ``A combined cohort of TCGA, TARGET and GTEx samples'' y es el resultado del preprocesamiento para poder unir las bases de datos que se describe en \cite{Vivian}.

\section{Motivación}

La motivación de este trabajo surge a partir del objetivo de detectar personas con tumores cerebrales de bajo grado y tumores cerebrales altamente agresivos de manera oportuna y certera a través de procesos que no impliquen una intervención quirúrgica.

La discusión de este trabajo está centrada en la interpretación de los resultados, las limitaciones de los métodos explorados así como de los datos utilizados, posibles mejoras dentro de los métodos explorados y las posibles implicaciones que tendría este trabajo para el planteamiento de futuros trabajos y protocolos de investigación.